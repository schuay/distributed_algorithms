\begin{Exc}{(A4.7.v0, 4 points):}
Prove that Algorithm 11 in the textbook [AW04] guarantees no deadlock.
\end{Exc}

We consider the two cases in which $p_0$, respectively, $p_1$
is in the entry section.

\begin{lemma} \label{lemma:1.p0loop}
If $p_0$ loops in Line 6, \lstinline|Want[0] = 1| and \lstinline|Want[1] = 1|.
\end{lemma}

\begin{proof}
Obvious from the code.
\end{proof}

\begin{lemma} \label{lemma:1.want1}
If \lstinline|Want[1] = 1|, $p_1$ is in lines 5-8 or in line 1.
\end{lemma}

\begin{proof}
Obvious from the code.
\end{proof}

\begin{lemma} \label{lemma:1.want0}
If \lstinline|Want[0] = 1|, $p_0$ is in lines 6-8.
\end{lemma}

\begin{proof}
Obvious from the code.
\end{proof}

\begin{lemma} \label{lemma:1.p0}
If $p_0$ is in the entry section, it eventually enters the critical section.
\end{lemma}

\begin{proof}
By contradiction. Assume $p_0$ is in the entry section in and subsequently
never enters the CS (critical section).
Since $p_0$ never enters the CS, it must loop in Line 6 forever, and
the variable assignments given in Lemma \ref{lemma:1.p0loop} must hold
whenever $p_0$ loops. By Lemma \ref{lemma:1.want1} and since 
\lstinline|Want[1] = 1|, $p_1$ can only be in

\begin{itemize}
\item Line 1, 5. In this case, \lstinline|Want[1]| is immediately set to 0 and $p_1$
      then loops in Line 2 until \lstinline|Want[0] = 0|. By Lemma \ref{lemma:1.want0},
      we know that this occurs only once $p_0$ passes Line 8, therefore
      it is ensured that $p_0$ observes \lstinline|Want[1] = 0| in Line 6 and
      enters the critical section.
\item Lines 6-8. Since no process may remain in the CS forever, \lstinline|Want[1]|
      is eventually set to 0 in Line 8. If it terminates in the remainder section,
      then $p_0$ trivially observes \lstinline|Want[1] = 0|. Otherwise, execution
      jumps to Line 1, in which case we may apply the argument of the previous point.      
\end{itemize}

Both cases lead to a contradiction, therefore $p_0$ eventually enters the critical
section.
\end{proof}

\begin{lemma} \label{lemma:1.p1}
If $p_1$ is in the entry section, a processor eventually enters the critical section.
\end{lemma}

\begin{proof}
Assume $p_1$ is in the entry section and observe that it
is prevented from entering the critical section only if \lstinline|Want[0] = 1|
whenever it is read in Line 2 and Line 5. In this case $p_0$ must be within Lines
6-8 (Lemma \ref{lemma:1.want0}). If $p_0$ is currently in Line 6, applying Lemma
\ref{lemma:1.p0} shows that $p_0$ must enter the CS. Otherwise, $p_0$ must
eventually execute the exit section; either $p_1$ then reads \lstinline|Want[0] = 0|
in Line 5 and enters the CS, or $p_0$ reenters the entry section and reaches the
CS by Lemma \ref{lemma:1.p0}.

\end{proof}


\begin{theorem}
Algorithm 11 guarantees no deadlock.
\end{theorem}

\begin{proof}
Immediate from Lemmas \ref{lemma:1.p0} and \ref{lemma:1.p1}.
\end{proof}

% ------------------------------------------------------------------------------

\begin{Exc}{(A4.14.v0, 2 points):}
Give a fast mutual exclusion algorithm using TAS variables and
show that it is correct.
\end{Exc}

\begin{algorithm}
\caption{TAS mutual exclusion} \label{alg:tas}
\begin{lstlisting}
/* Initially, locked = 0. */
/* Entry section. */
while (tas(locked) == 1) { /* Do nothing. */ }
/* Critical section. */
/* Exit section. */
reset(locked);
/* Remainder section. */
\end{lstlisting}
\end{algorithm}

% TODO

% ------------------------------------------------------------------------------

\newcommand{\last}[1]{#1.\mathit{last}}
\newcommand{\first}[1]{#1.\mathit{first}}
\newcommand{\VV}[1]{\langle #1 \rangle}
\newcommand{\pos}{\mathit{pos}}
\newcommand{\queue}{\mathit{queue}}
\newcommand{\RMW}{\mathit{RMW}}

\begin{Exc}{(L.10.9.v2, 11 points):}

Consider the mutual exclusion algorithm with RMW variables
(Algorithm 8) from the textbook. The code for every processor
reads:

\begin{code}%
\vspace*{-1cm}%
Initially $V=\VV{0,0}$\\
\\
\emn{/* Code for entry section: */}\NL
\> $\pos := \RMW(V,\VV{\first{V},\last{V}+1})$\emn{\ \ \ // enqueueing at tail}\NL
\> repeat\NL
\>\> $\queue := \RMW(V,V)$\emn{\ \ \ // read head of queue}\NL
\> until $\first{\queue} = \last{\pos}$\emn{\ \ \ // until becomes first}\\
\\
\emn{/* Critical section */}\\
\\
\emn{/* Code for exit section */}\NL
\> $\RMW(V,\VV{\first{V}+1,\last{V}})$\emn{\ \ \ // advance head of queue}
\end{code}

For the version of the Algorithm 8' where all variables have unbounded
range, give a detailed proof that it guarantees mutual exclusion and
0-bounded waiting [+ no deadlock]. Moreover, prove that always
$\first{V} \leq \last{V} \leq \first{V}+n$.
\end{Exc}

This exercise consists of filling in proofs for several lemmas from the 
slides\footnote{Lemmas from lecture slides are distinguished from lemmas in this
homework by appending the prime character.}:
Lemma $166'$ (Lemma \ref{lemma:3.mutual_exclusion}) for mutual exclusion,
and Lemma $167'$ (Lemma \ref{lemma:3.0bounded}) for 0-bounded waiting. 
$\first{V} \leq \last{V} \leq \first{V}+n$ follows directly from Lemma $167'$.

\begin{lemma} \label{lemma:3.mutual_exclusion}
In every reachable configuration of Algorithm $8'$, there is at most one process
$p_i$ within the $\text{critical } \cup \text{exit}$ section (CES), and its ticket
satisfies $\last{\pos_i} = \first{V}$.
\end{lemma}

\begin{proof}
By induction on the configuration $C_i$.

In configuration $C_0$ the lemma is trivially satisfied since no process is in
the CES.

Assume that in configuration $C_{k-1}$ there is at most one process
$p_i$ within the CES, and its ticket
satisfies $\last{\pos_i} = \first{V}$.

We now examine the step $(C_{k-1}, \phi_k, C_k)$ and distinguish between two cases: 
\begin{enumerate}
\item \emph{Processor $p_i$ is in the CES in $C_{k-1}$.} We must show that in $\phi_k$,
      no $p_{j \neq i}$ enters the CES, and that if $p_i$
      remains in the CES, $\last{\pos_i}$ and $\first{V}$ do not change.
      \begin{enumerate}[a)]
      \item \emph{No $p_{j \neq i}$ enters the CES.} 
            Since a processor $p_j$ only enters the CES
            if $\last{\pos_j} = \first{V}$ (by the code),
            $\last{\pos_i} = \first{V}$ in $C_{k-1}$ (induction hypothesis),
            and each processor draws a unique ticket $\last{\pos}$ (code and
            semantics of $\RMW$), no $p_{j \neq i}$ can enter the CES in $\phi_k$.
      \item \emph{If $p_i$ remains in the CES, $\last{\pos_i}$ and $\first{V}$ 
            do not change.} $\last{\pos_i}$ is a local variable and thus cannot
            be accessed by any processor other than $p_i$, and is not modified 
            in the CES by $p_i$ itself. From the code, we see that $\first{V}$
            is only modified in the exit section. However, since there is no
            other processor $p_{j \neq i}$ in the CES in $C_{k-1}$
            (induction hypothesis) and $p_i$ itself stays in the CES in $\phi_k$,
            no processor can execute Line 6 in $\phi_k$, hence $\first{V}$
            remains unchanged.            
      \end{enumerate}
\item \emph{There is no processor in the CES in $C_{k-1}$.}
      In this case, it only remains to show that if a processor $p_i$ enters
      the CS in $\phi_k$, $\last{\pos_i} = \first{V}$. This is trivially true,
      since $p_i$ may only enter the the CES if $\phi_k$ reads
      $\last{\pos_i} = \first{V}$, and neither $\last{\pos_i}$ nor $\first{V}$
      are modified in $\phi_k$
\end{enumerate}
Thus it must also hold in configuration $C_k$ that there is at most one process
$p_i$ within the $\text{critical } \cup \text{exit}$ section, and its ticket
satisfies $\last{\pos_i} = \first{V}$
\end{proof}


\begin{lemma} \label{lemma:3.0bounded}
In every reachable configuration of Algorithm $8'$, every processor $p_i$
that is within the $\text{entry } \cup \text{critical } \cup \text{exit}$
section (ECES) has drawn (and not returned) a unique ticket $\last{\pos_i}$ in the
interval $[\first{V}, \last{V})$, and $\last{V} - \first{V}$ equals the number
$d$ of processors that have drawn a ticket.
\end{lemma}

\begin{proof}
By induction on the configuration $C_i$.

In configuration $C_0$ the lemma is trivially satisfied since no process is in
the ECES and $V = \VV{0, 0}$.

Assume that in configuration $C_{k-1}$ every processor $p_i$
that is within the ECES has drawn (and not returned) a unique ticket $\last{\pos_i}$ in the
interval $[\first{V}, \last{V})$, and $\last{V} - \first{V}$ equals the number
$d$ of processors that have drawn a ticket.

We now examine the step $(C_{k-1}, \phi_k, C_k)$. Note that a tickets is
drawn (returned) in $phi_k$ iff $\last{V}$ ($\first{V}$) is modified in $\phi_k$.
The ECES is entered when a ticket is drawn, and left when it is returned.
Let the value of $\last{V}$ ($\first{V}$) in configuration $C_k$ be denoted
$\last{V_k}$ ($\first{V_k}$).

\begin{itemize}
\item If $\phi_k$ does not modify $V$, then no processor draws or returns a ticket,
      and no processor enters or leaves the ECES, and thus the induction hypothesis (IH)
      trivially holds in $C_k$.
\item If $\last{V}$ is modified in $\phi_k$ by some processor $p_i$, then $p_i$ has executed
      Line 2 and has entered the ECES in $C_k$. Both the number of drawn tickets
      and $\last{V}$ have increased by one in $C_k$, hence $\last{V_k} - \first{V_k} = drawn\_tickets$
      holds. Uniqueness of $p_i$'s newly drawn ticket is given by the fact that
      the value of $p_i$'s ticket is $\last{V_{k-1}}$ while all tickets in $C_{k-1}$
      were in the range $[\first{V_{k-1}}, \last{V_{k-1}})$. Finally, since the ticket domain
      is expanded by one in $\phi_k$ (and not shrunk), all tickets that were previously
      in the ticket domain in $C_{k-1}$ are still within the ticket domain in $C_k$.
      $p_i$'s new ticket value $\last{V_{k-1}} = \last{V_k} - 1$ is trivially within
      the ticket domain in $C_k$.
\item If $\first{V}$ is modified in $\phi_k$ by some processor $p_i$, then $p_i$ has executed
      Line 6 and left the ECES in $C_k$. Both the number of drawn tickets
      and $\last{V}$ have decreased by one in $C_k$, hence
      $\last{V_k} - \first{V_k} = drawn\_tickets$ holds.
      By Lemma \ref{lemma:3.mutual_exclusion}, $p_i$'s ticket equals $\first{V_{k-1}}$;
      since $\first{V_k} = \first{V_{k-1} + 1}$ and all drawn tickets are unique
      and within the ticket domain $[\first{V_{k-1}}, \last{V_{k-1}})$ in $C_{k-1}$ (IH),
      all drawn tickets are still within the ticket domain $[\first{V_k}, \last{V_k})$
      in $C_k$.
\end{itemize}

\end{proof}

\begin{theorem}
Algorithm $8'$ guarantees mutual exclusion and 0-bounded waiting.
\end{theorem}

\begin{proof}
See Lemmas \ref{lemma:3.mutual_exclusion} and \ref{lemma:3.0bounded} together with
Theorems $166'$ and $167'$ in the slides.
\end{proof}


\begin{lemma} \label{lemma:3.Vbounds}
In every reachable configuration of Algorithm $8'$,
$\first{V} \leq \last{V} \leq \first{V}+n$ holds.
\end{lemma}

\begin{proof}
It follows directly from that code that no processor ever holds more than one
ticket. Since there are $n$ processors, Lemma \ref{lemma:3.0bounded} gives us
both the lower bound for $\last{V}$ with $0$ drawn tickets: $\first{V} = \last{V}$;
and the upper bound with $n$ drawn tickets: $\first{V} + n = \last{V}$.
\end{proof}


% ------------------------------------------------------------------------------

\begin{Exc}{(L10.44.1.v0, 5 points):}
Recall the proof that every $k$-bounded waiting $n$-processor mutual exclusion
algorithm with RMW variables requires at least $n$ different shared
memory states from the textbook. This proof cannot be generalized to apply also to no-lockout
$n$-processor mutual exclusion algorithms in general, for the reason that one must
append a schedule $\sigma$ to the ``offending'' schedule $\omega=\tau_0\tau_1\dots\tau_j\rho$
(where $p_{\ell}$ overtakes $p_j$ $k+1$ times), which lets all processors
$p_0,\dots,p_j$ take infinitely many steps. Otherwise, the resulting schedule
would not lead to an admissible execution. However, in $\sigma$, $p_j$ may of course enter the
critical section, which would secure no lockout in $\omega\sigma$.

However, if we restrict our attention to no lockout $n$-processor
mutual exclusion algorithms with RMW variables, which satisfy the additional
property that every fair execution of processors $p_0,\dots,p_i$
(i.e., where all these processes make infinitely many steps)
visits all the [finitely many] possible SHM states infinitely often,
the proof can be
made to work: Prove that any such algorithm requires at least $n$
different shared memory states.
\end{Exc}
