\begin{Exc}{(A4.7.v0, 4 points):}
Prove that Algorithm 11 in the textbook [AW04] guarantees no deadlock.
\end{Exc}

% ------------------------------------------------------------------------------

\begin{Exc}{(A4.14.v0, 2 points):}
Give a fast mutual exclusion algorithm using TAS variables and
show that it is correct.
\end{Exc}

% ------------------------------------------------------------------------------

\begin{Exc}{(L.10.9.v2, 11 points):}
\newcommand{\last}[1]{#1.\mathit{last}}
\newcommand{\first}[1]{#1.\mathit{first}}
\newcommand{\VV}[1]{\langle #1 \rangle}
\newcommand{\pos}{\mathit{pos}}
\newcommand{\queue}{\mathit{queue}}
\newcommand{\RMW}{\mathit{RMW}}

Consider the mutual exclusion algorithm with RMW variables
(Algorithm 8) from the textbook. The code for every processor
reads:

\begin{code}%
\vspace*{-1cm}%
Initially $V=\VV{0,0}$\\
\\
\emn{/* Code for entry section: */}\NL
\> $\pos := \RMW(V,\VV{\first{V},\last{V}+1})$\emn{\ \ \ // enqueueing at tail}\NL
\> repeat\NL
\>\> $\queue := \RMW(V,V)$\emn{\ \ \ // read head of queue}\NL
\> until $\first{\queue} = \last{\pos}$\emn{\ \ \ // until becomes first}\\
\\
\emn{/* Critical section */}\\
\\
\emn{/* Code for exit section */}\NL
\> $\RMW(V,\VV{\first{V}+1,\last{V}})$\emn{\ \ \ // advance head of queue}
\end{code}

For the version of the Algorithm 8' where all variables have unbounded
range, give a detailed proof that it guarantees mutual exclusion and
0-bounded waiting [+ no deadlock]. Moreover, prove that always
$\first{V} \leq \last{V} \leq \first{V}+n$.
\end{Exc}

% ------------------------------------------------------------------------------

\begin{Exc}{(L10.44.1.v0, 5 points):}
Recall the proof that every $k$-bounded waiting $n$-processor mutual exclusion
algorithm with RMW variables requires at least $n$ different shared
memory states from the textbook. This proof cannot be generalized to apply also to no-lockout
$n$-processor mutual exclusion algorithms in general, for the reason that one must
append a schedule $\sigma$ to the ``offending'' schedule $\omega=\tau_0\tau_1\dots\tau_j\rho$
(where $p_{\ell}$ overtakes $p_j$ $k+1$ times), which lets all processors
$p_0,\dots,p_j$ take infinitely many steps. Otherwise, the resulting schedule
would not lead to an admissible execution. However, in $\sigma$, $p_j$ may of course enter the
critical section, which would secure no lockout in $\omega\sigma$.

However, if we restrict our attention to no lockout $n$-processor
mutual exclusion algorithms with RMW variables, which satisfy the additional
property that every fair execution of processors $p_0,\dots,p_i$
(i.e., where all these processes make infinitely many steps)
visits all the [finitely many] possible SHM states infinitely often,
the proof can be
made to work: Prove that any such algorithm requires at least $n$
different shared memory states.
\end{Exc}
