\newcommand{\off}[1]{\text{off}_{#1}}
\newcommand{\rev}[1]{\text{rev}(#1)}

\begin{Exc}{(A3.9.v0, 4 points):}
Consider the ring $R_n^{rev}$, where processor $p_i$ has UID
$rev(i)$, $0\leq i \leq n-1$ for $n=2^k$. Prove that all
consecutive segments of size $2^\ell$, $0<\ell < k$, in $R_n^{rev}$
are order-equivalent.

\normalfont
Hint: For any $i'=i \mod 2^\ell$ and $j'=j \mod 2^\ell$ with $j\neq i$,
show that $rev(i) - rev(j)>0$ implies $rev(i') - rev(j')>0$.
\end{Exc}

\begin{proof}
Take any index $0 \leq s < n$, any $0 < \ell < k$, and any $0 \leq \off{i}, \off{j} < 2^\ell$ (all
arithmetic in this proof is modulo $n$).
Let indices $i = s + \off{i}$, $j = s + \off{j}$, $i' = i + 2^\ell$, and $j' = j + 2^\ell$, i.e.
we consider the segment $S$ of length $2^\ell$ starting at $s$ and its neighbor segment $S'$ starting
at $s' = s + 2^\ell$. $i, j$ are arbitrary nodes within $S$, and $i', j'$ their corresponding nodes
in $S'$.

We now prove that $S, S'$ are order-equivalent by showing that
$\rev{i} < \rev{j} \iff \rev{i'} < \rev{j'}$.

Notice that a) adding $2^\ell$ to $i$ ($j$) does not affect the $\ell$ least significant bits of $i$ ($j$),
and that b) in any sequence of $2^\ell$ consecutive numbers, the $\ell$ least significant bits are unique.
It follows from a) that the $\ell$ least significant bits of $i'$ ($j'$) are identical to those of
$i$ ($j$). These in turn are the $\ell$ most significant bits in $\rev{i}, \rev{j}, \rev{i'}, \rev{j'}$,
and it follows from b) that they completely determine the ordering relation of all elements in segments
$S, S'$.
\end{proof}

% ------------------------------------------------------------------------------

\begin{Exc}{(A3.10.1.v1, 5 points):}
Consider an anonymous ring where processors start with binary
inputs. Prove rigorously that there is no uniform synchronous
algorithm for computing the AND of all the input bits.

\normalfont
Hint: Consider executions in rings $R$ with all-``1'' inputs and
in large rings $R'$ with a single``0'' input.
\end{Exc}

\begin{proof}
Fix any anonymous, uniform, synchronous algorithm $A$ that computes the logical AND over the input of
all processors. Assume that it terminates with result $1$ in $r(n)$ rounds when executed on a ring $R$ consisting
of $n$ processors $p_0, \ldots, p_{n-1}$ with all inputs equalling $1$.

We now construct a ring $R'$ by combining a segment of $n + 2r(n)$ processors with input $1$
($\underline{p}_i, p_i, \overline{p}_i$) with a single processor with input $0$ ($q$):
$\underline{p}_0, \ldots, \underline{p}_{r(n) - 1},
 p_0, \ldots, p_{n-1},
 \overline{p}_0, \ldots, \overline{p}_{r(n) - 1},
 q.$
Consider the first $r(n)$ rounds of an execution of $A$ on the ring $R'$. Since by anonymity
the $r(n)$-neighborhood
of each $p_i$ in $R'$ is equivalent to the $r(n)$ neighborhood for each $p_i$ in $R$, their state
after $r(n)$ rounds must be identical, i.e. they must have terminated with result $1$. This is however
a contradiction, since input $0$ of node $q$ forces the output of the algorithm to $0$.
\end{proof}

% ------------------------------------------------------------------------------

\begin{Exc}{(A3.10.3.v3, 7 points):}
Consider an anonymous ring where processors start with binary
inputs. Prove that $\Omega(n^2)$ is a lower bound for the
message complexity of any (non-uniform, message-driven)
asynchronous algorithm that computes the AND of all the
input bits. Show by means of a counter-example that this
lower bound does not hold for synchronous algorithms.

\normalfont
Hint: Consider a synchronous (lockstep round) execution of the asynchronous
algorithm, and use similar arguments as in the proof of the message complexity lower bound
for synchronous comparison-based leader election: What can you say about the state of $p_i$ after
round $k$? What is a suitable choice for a ``highly symmetric'' ring here? How many
(active) rounds are at least required to compute the AND correctly?
\end{Exc}

\begin{theorem}
The lower bound of $\Omega(n^2)$ messages for an anonymous, non-uniform algorithm calculating the
logical AND of all binary inputs does not hold in synchronous systems.
\end{theorem}

\begin{proof}
The following algorithm for a ring of size $n$ with processors $p_0, \ldots, p_{n-1}$
is given as a counterexample. Let the maximum over all processor ids be $p_{max}$.
The algorithm proceeds in $p_{max}$ phases of $n$ rounds each.
If it has not terminated yet, processor $p_i$ sends a leader message containing its id
to its left neighbor and terminates as the leader. Upon receiving a message, non-terminated processors
terminate as non-leaders and forward the message to their left neighbor.

This algorithm elects a leader with $O(n)$ message complexity (only the processor with the lowest
id gets to send its leader message, which is then forwarded exactly once by all other processors in
the ring).
\end{proof}


% ------------------------------------------------------------------------------

\begin{Exc}{(S2.3.v0, 10 points):}
Consider the simple asynchronous leader election algorithm of Sec.~3.3.1 of
the textbook (Slide 82 of the lecture slides). Provide the complete
pseudo-code description and the transition relation of this algorithm,
and a complete and rigorous correctness proof (safety and liveness).

\normalfont
Hint: Try to find a suitable invariant for your safety proof, which allows
you to deduce that non-maximal $id_i$'s can never make it around the ring.
The liveness proof could use the proof of the flooding algorithm
(Algorithm 2) as an inspiration.
\end{Exc}
