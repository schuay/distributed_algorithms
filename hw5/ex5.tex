\begin{Exc}{(S.6.1b.v0, 3 points):}
Prove or disprove: The intersection cut $\vec{I} = \vec{C}_1 \cap \vec{C}_2$
of two consistent cuts $\vec{C}_1$ and $\vec{C}_2$ in the same
execution $\alpha$, defined as the
component-wise minimum $I[k]=\min\{C_1[k],C_2[k]\}$ for all $k$,
is consistent.
\end{Exc}
\begin{Exc}{(S.6.10.v0, 5 points):}
Prove or disprove: The Chandy-Lamport Snapshot Algorithm 19 from
the textbook also works correctly when more than one processor gets
an initial indication to take a snapshot.

Same question if channel states [messages in transit from inside
the cut] are also recorded in Algorithm 19.
\end{Exc}
\begin{Exc}{(S.6.15.v0, 8 points):}
Given any real $c \geq 1/2$ and message delay uncertainty $u$ in a system
of $n>2$ processors, prove that clock
synchronization with skew $\epsilon=c\cdot u$ requires a communication graph
with diameter $D \leq 2c$.

\normalfont
Hint: Devise a contradiction proof, using a shifting argument for
processors $p$, $q$ at maximum distance $D>1$: By shifting all
processors at distance $k$ from $p$ by a certain amount in a
shifted execution $\alpha'$, obtained from some suitable ``average''
execution $\alpha$, $q$ can be shifted considerably both
left and right. Use the technique introducted in the lecture
to show that this does not allow the skew produced by any
deterministic clock synchronization algorithm ${\cal A}$
to attain $c\cdot u$.
\end{Exc}
